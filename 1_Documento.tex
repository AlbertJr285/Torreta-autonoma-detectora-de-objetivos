%%%%%%%%%%%%%%%%%%%%%%%%%%%%%%%%%%%%%%%%%%%%%%%%%%%%%%%%%%%%%%%%%%%%%%%%%
%%%%%%%%%%%%%%%%%%%%%%%%%%%%%%%%%%%%%%%%%%%%%%%%%%%%%%%%%%%%%%%%%%%%%%%%%
%%%                                                                   %%%
%%%  PLANTILLA DE TRABAJOS DE GRADO UNIVERSIDAD CATÓLICA BOLIVIANA    %%% 
%%%                                                                   %%%
%%%       AUTOR PRINCIPAL TEMPLATE:   RODOLFO JESÚS PRIETO MALDONADO  %%%
%%%       E-MAIL                  :   rodolfo.jp.13@gmail.com         %%%
%%%       AUTOR GUÍA/APOYO        :   EDWIN CALLA DURANDAL            %%%
%%%       E-MAIL                  :   ecd@ucbcba.edu.bo               %%%
%%%                                   edwin.calla@gmail.com           %%%
%%%       GESTIÓN ELABORACIÓN     :   2 - 2018                        %%%
%%%     ÚLTIMA ACTUALIZACIÓN      :   20 de agosto de 2019            %%%
%%%                                    Horas 19:00                    %%%
%%%                                                                   %%%
%%%%%%%%%%%%%%%%%%%%%%%%%%%%%%%%%%%%%%%%%%%%%%%%%%%%%%%%%%%%%%%%%%%%%%%%%
%%%%%%%%%%%%%%%%%%%%%%%%%%%%%%%%%%%%%%%%%%%%%%%%%%%%%%%%%%%%%%%%%%%%%%%%%
%
% Junto a las modalidades de titulación y el objetivo que persigue la carrera de Ing. Mecatrónica de formar profesionales íntegros y de alto nivel, se dispone al estudiante el Template de Latex para la elaboración del documento de titulación de la carrera, como también para presentar cualquier tipo de trabajo dentro de la misma.
%
% La presente plantilla se encuentra disponible en la página web:
% https://www.ucbcba.edu.bo/depto-ingenieria-y-cs-exactas/ingenieria-mecatronica/
% En la pestaña Template Latex.     
% 
% IMPORTANTE
% Los espacios a ser modificados serán identificados de la siguiente forma:

%------------------------
%
%       DEDICATORIA
%
%------------------------

% Los demás espacios y códigos son parte del formato, si se desea modificar los mismos, es bajo la responsabilidad del editor, siendo que podrá alterar el formato de la institución. 
% Caso usted encuentre algún problema de formato u otro tipo, usted encontrará un formulario para reportar cualquier bug (error) en la página web donde se descargo el presente Template. 

% Para comenzar a editar y trabajar, por favor dirigirse a la línea de código del presente archivo número 454 y llenar los espacios necesarios, posterior a esto, dirigirse a dedicatoria, resumen, introducción, marco teórico, marco metodológico, conclusiones, anexos y apéndices. Usted podrá modificar los mismos según sus necesidades. 

%%%%%% NO MODIFICAR %%%%%%% NO MODIFICAR %%%%%%%%%% NO MODIFICAR %%%%%%%%% NO MODIFICAR %%%%%%%%
%%%%%%%%%%%%%%% NO TOCAR %%%%%%%%%%%%% NO TOCAR %%%%%%%%%%%%% NO TOCAR %%%%%%%%%% NO TOCAR %%%%%
%%%%%% NO MODIFICAR %%%%%%%% NO MODIFICAR %%%%%%%%%% NO MODIFICAR %%%%%%% NO MODIFICAR %%%%%%%%%
%%%%%%%%%%% NO TOCA R%%%%%%%%%%%%% NO TOCAR %%%%%%%%%%%%% NO TOCAR %%%%%%%%%%%%%% NO TOCAR %%%%%

% Inicio código formato
%   Condiciones generales   |   tamaño de hoja, tamaño de letra en general
\documentclass[letterpaper,12pt]{article}   % carta, letra 12

%---------- PAQUETES --------------

%  Márgenes
\usepackage[left=3.5cm,right=3cm,top=3cm,bottom=3cm]{geometry}
%% Básicos
\usepackage[T1]{fontenc}            % Letras vectoriales (mejor calidad)
\usepackage[utf8]{inputenc}         % Habilita caracteres extra como acentos
\usepackage[spanish]{babel}         % Idioma (silabación y gestión de palabras)
\usepackage{csquotes}               % Reglas del lenguaje
\usepackage{mathptmx}               % Fuente +parecida a 'Times New Roman'
\usepackage{enumitem}               % Habilita enumeración
\usepackage{fancyhdr}               % Cabeceras y pies de página 
\usepackage[table,xcdraw]{xcolor}   % Habilita cambios de color (texto, hoja, etc)
\usepackage{titling}                % Títulos y nombres en el documento
\usepackage{titlesec}               % Edición de formato de secciones
\usepackage{tocloft}                % Índice
\usepackage{setspace}               % Interlineado

\usepackage[linktocpage=true,hidelinks]{hyperref}           % Habilita hipervínculos 
                                                            % quitar pagebackref para imprimir
\usepackage{multicol}                   % MULTIPLE COLUMNAS
\usepackage{multirow}                   % MULTIPLES FILAS
\usepackage[absolute]{textpos}
\usepackage[acronym,xindy]{glossaries}
%\usepackage{glossary-mcols}
\usepackage{array}
\usepackage{makecell} %eol on tables

\usepackage{svg}
\usepackage[outline]{contour}
\usepackage[letterspace=40]{microtype}
\usepackage{xifthen}                            % provides \isempty test

%% Matemáticos
\usepackage{amsmath, amsthm, amssymb, amsfonts, upgreek}
\spanishdecimal{.}

%% Pseudo algoritmos
\usepackage[]{algorithmic,algorithm}
\renewcommand{\algorithmicrequire}{\textbf{Inicio}}
\renewcommand{\algorithmicensure}{\textbf{Fin}}
\usepackage[algo2e,boxed,linesnumbered,ruled,vlined]{algorithm2e}
\floatname{algorithm}{Algoritmo}

%% Código Matlab
\usepackage{listings}
\lstset{language=Matlab, breaklines=true, basicstyle=\footnotesize}
\lstset{numbers=left, numberstyle=\tiny, stepnumber=1, numbersep=-2pt}

%% Imágenes
\usepackage{graphicx}           % Inclusión de imágenes
\usepackage{wrapfig}            % Optimiza texto alrededor de las imágenes
\usepackage{float}              % Ubicación inteligente de imágenes 
\usepackage{caption}            % Habilita edición de títulos de figuras
\usepackage{subfig}             % Habilita edición de sub-figuras

\usepackage{pdfpages}           % Introduce pdf (anexos)
\usepackage{longtable}


%% Extras
\usepackage{verbatim}           % Habilita big-comment y ambientes especiales
\usepackage{lipsum} 
\usepackage{array}%\usepackage{multirow}
\usepackage{afterpage}          % obliga a una tabla a estar en su lugar
                                %   \afterpage{\clearpage}  usa esto antes de la tabla

% Gestión de bibliografía
\usepackage[backend=biber, style=apa]{biblatex}
\DeclareLanguageMapping{spanish}{spanish-apa}
\addbibresource{bibliografia.bib}

%%%%%% NO MODIFICAR %%%%%%% NO MODIFICAR %%%%%%%%%% NO MODIFICAR %%%%%%%%% NO MODIFICAR %%%%%%%%
%%%%%%%%%%%%%%% NO TOCAR %%%%%%%%%%%%% NO TOCAR %%%%%%%%%%%%% NO TOCAR %%%%%%%%%% NO TOCAR %%%%%
%%%%%% NO MODIFICAR %%%%%%%% NO MODIFICAR %%%%%%%%%% NO MODIFICAR %%%%%%% NO MODIFICAR %%%%%%%%%
%%%%%%%%%%% NO TOCA R%%%%%%%%%%%%% NO TOCAR %%%%%%%%%%%%% NO TOCAR %%%%%%%%%%%%%% NO TOCAR %%%%%

%------- FORMATO ------------------

%grosor fijo de las columnas en las tablas
\newcolumntype{L}[1]{>{\raggedright\let\newline\\\arraybackslash\hspace{0pt}}m{#1}}
\newcolumntype{C}[1]{>{\centering\let\newline\\\arraybackslash\hspace{0pt}}m{#1}}
\newcolumntype{R}[1]{>{\raggedleft\let\newline\\\arraybackslash\hspace{0pt}}m{#1}}
\newcolumntype{J}[1]{>{\let\newline\\\arraybackslash\hspace{0pt}}m{#1}}

% eol onsie tables
\renewcommand\theadalign{bc}
\renewcommand\theadfont{\bfseries}
\renewcommand\theadgape{\Gape[4pt]}
\renewcommand\cellgape{\Gape[4pt]}

%%%%%% NO MODIFICAR %%%%%%% NO MODIFICAR %%%%%%%%%% NO MODIFICAR %%%%%%%%% NO MODIFICAR %%%%%%%%
%%%%%%%%%%%%%%% NO TOCAR %%%%%%%%%%%%% NO TOCAR %%%%%%%%%%%%% NO TOCAR %%%%%%%%%% NO TOCAR %%%%%
%%%%%% NO MODIFICAR %%%%%%%% NO MODIFICAR %%%%%%%%%% NO MODIFICAR %%%%%%% NO MODIFICAR %%%%%%%%%
%%%%%%%%%%% NO TOCA R%%%%%%%%%%%%% NO TOCAR %%%%%%%%%%%%% NO TOCAR %%%%%%%%%%%%%% NO TOCAR %%%%%

%--------Formato de Índices---------

\renewcommand{\cftsecleader}{\cftdotfill{\cftdotsep}}
\renewcommand{\cftfigleader}{\cftdotfill{\cftdotsep}}
\renewcommand{\cfttableader}{\cftdotfill{\cftdotsep}}

\renewcommand{\cftsecaftersnum}{.}%
\renewcommand{\cftsubsecaftersnum}{.}%
\renewcommand{\cftsubsubsecaftersnum}{.}%
\renewcommand{\cftparaaftersnum}{.}%
\renewcommand{\cftsubparaaftersnum}{.}%
    
\renewcommand{\cfttoctitlefont}{\hspace*{\fill}\normalsize\bf\MakeUppercase}
\renewcommand{\cftaftertoctitle}{\hspace*{\fill}}
\renewcommand{\cftlottitlefont}{\hspace*{\fill}\normalsize\bf\MakeUppercase}
\renewcommand{\cftafterlottitle}{\hspace*{\fill}}
\renewcommand{\cftloftitlefont}{\hspace*{\fill}\normalsize\bf\MakeUppercase}
\renewcommand{\cftafterloftitle}{\hspace*{\fill}}

\renewcommand{\cftbeforesecskip}{2pt}
\renewcommand{\cftbeforesubsecskip}{-6pt}
\renewcommand{\cftbeforesubsubsecskip}{-6pt}
\renewcommand{\cftbeforeparaskip}{-6pt}
\renewcommand{\cftbeforesubparaskip}{-6pt}
\renewcommand{\cftbeforefigskip}{-6pt}
\renewcommand{\cftbeforetabskip}{-6pt}

\renewcommand{\cftsecfont}{\bfseries}
\renewcommand{\cftsubsecfont}{\bfseries}
\renewcommand{\cftsubsubsecfont}{\bfseries\itshape}
\renewcommand{\cftparafont}{\itshape}
\renewcommand{\cftsubparafont}{\itshape}

\renewcommand{\cftsecpagefont}{}

\cftsetindents{figure}{0em}{5em}
\renewcommand{\cftfigpresnum}{\bfseries Figura }

\cftsetindents{table}{0em}{4.6em}
\renewcommand{\cfttabpresnum}{\bfseries Tabla }

%-----------Referenciación-------------------------
\hypersetup{
    colorlinks=false, %set true if you want colored links
    linktoc=all,      %set to all if you want both sections and subsections linked
    linkcolor=black,  %choose some color if you want links to stand out
}

%%%%%% NO MODIFICAR %%%%%%% NO MODIFICAR %%%%%%%%%% NO MODIFICAR %%%%%%%%% NO MODIFICAR %%%%%%%%
%%%%%%%%%%%%%%% NO TOCAR %%%%%%%%%%%%% NO TOCAR %%%%%%%%%%%%% NO TOCAR %%%%%%%%%% NO TOCAR %%%%%
%%%%%% NO MODIFICAR %%%%%%%% NO MODIFICAR %%%%%%%%%% NO MODIFICAR %%%%%%% NO MODIFICAR %%%%%%%%%
%%%%%%%%%%% NO TOCA R%%%%%%%%%%%%% NO TOCAR %%%%%%%%%%%%% NO TOCAR %%%%%%%%%%%%%% NO TOCAR %%%%%

%-------------Comandos de citación-----------------------
\renewbibmacro*{cite:labelyear+extrayear}{%
\iffieldundef{labelyear}
{}
{\printtext[bibhyperref]{%
\printfield{labelyear}%
\printfield{extrayear}}}}

\newbibmacro*{cite:authoryear}{%
\printnames[][-\value{listtotal}]{labelname}
\setunit*{\printdelim{nameyeardelim}}%
\printtext[bibhyperlink]{%
\usebibmacro{cite:labelyear}}}

\newbibmacro*{cite:authoryear2}{%
Cf. \printnames[][-\value{listtotal}]{labelname}
\setunit*{\printdelim{nameyeardelim}}%
\printtext[bibhyperlink]{%
\usebibmacro{cite:labelyear}}}

\newbibmacro*{cite:authoryear3}{%
\printnames[][-\value{listtotal}]{labelname}
\printtext[bibhyperlink]{%
(\usebibmacro{cite:labelyear}}}

\newbibmacro*{cite:authoryear4}{%
\printnames[][-\value{listtotal}]{labelname}
\printtext[bibhyperlink]{%
\usebibmacro{cite:labelyear}}}

\DeclareCiteCommand{\cite}[]
  {\usebibmacro{prenote}}
  {\usebibmacro{citeindex}%
   \printtext[bibhyperref]{\usebibmacro{cite:authoryear3}}}
  {\multicitedelim}
  {\usebibmacro{postnote})}

\DeclareCiteCommand*{\cite}[]
  {\usebibmacro{prenote}}
  {\usebibmacro{citeindex}%
   \printtext[bibhyperref]{\usebibmacro{cite:authoryear3}}}
  {\multicitedelim}
  {\usebibmacro{postnote}}
  
  \DeclareCiteCommand{\citel}[]
  {\usebibmacro{prenote}}
  {\usebibmacro{citeindex}%
   \printtext[bibhyperref]{\usebibmacro{cite:authoryear4}}}
  {\multicitedelim}
  {\usebibmacro{postnote}}

\DeclareCiteCommand*{\citel}[]
  {\usebibmacro{prenote}}
  {\usebibmacro{citeindex}%
   \printtext[bibhyperref]{\usebibmacro{cite:authoryear4}}}
  {\multicitedelim}
  {\usebibmacro{postnote}}

\DeclareCiteCommand{\citep}[\mkbibparens]
  {\usebibmacro{prenote}}
  {\usebibmacro{citeindex}%
   \printtext[bibhyperref]{\usebibmacro{cite:authoryear}}}
  {\multicitedelim}
  {\usebibmacro{postnote}}

\DeclareCiteCommand*{\citep}[\mkbibparens]
  {\usebibmacro{prenote}}
  {\usebibmacro{citeindex}%
   \printtext[bibhyperref]{\usebibmacro{cite:authoryear}}}
  {\multicitedelim}
  {\usebibmacro{postnote}}
  
  \DeclareCiteCommand{\citecf}[\mkbibparens]
  {\usebibmacro{prenote}}
  {\usebibmacro{citeindex}%
   \printtext[bibhyperref]{\usebibmacro{cite:authoryear2}}}
  {\multicitedelim}
  {\usebibmacro{postnote}}

\DeclareCiteCommand*{\citecf}[\mkbibparens]
  {\usebibmacro{prenote}}
  {\usebibmacro{citeindex}%
   \printtext[bibhyperref]{\usebibmacro{cite:authoryear2}}}
  {\multicitedelim}
  {\usebibmacro{postnote}}
  
    \renewcommand*{\postnotedelim}{\addcolon\addspace}
    \DeclareFieldFormat{postnote}{#1}
    \DeclareFieldFormat{multipostnote}{#1}

%%%%%% NO MODIFICAR %%%%%%% NO MODIFICAR %%%%%%%%%% NO MODIFICAR %%%%%%%%% NO MODIFICAR %%%%%%%%
%%%%%%%%%%%%%%% NO TOCAR %%%%%%%%%%%%% NO TOCAR %%%%%%%%%%%%% NO TOCAR %%%%%%%%%% NO TOCAR %%%%%
%%%%%% NO MODIFICAR %%%%%%%% NO MODIFICAR %%%%%%%%%% NO MODIFICAR %%%%%%% NO MODIFICAR %%%%%%%%%
%%%%%%%%%%% NO TOCA R%%%%%%%%%%%%% NO TOCAR %%%%%%%%%%%%% NO TOCAR %%%%%%%%%%%%%% NO TOCAR %%%%%

%---------------Formato de títulos------------------------------------
\titleformat{\section}
{\normalsize\bfseries}{\thesection.}{1em}{}

\titleformat{\subsection}
{\normalsize\bfseries}{\thesubsection.}{1em}{}

\titleformat{\subsubsection}
{\normalsize\bfseries\itshape}{\thesubsubsection.}{1em}{}

\setcounter{tocdepth}{4}%
\setcounter{secnumdepth}{5}%

\titleformat{\paragraph}
{\normalsize\itshape}{\theparagraph.}{1em}{}

\titleformat{\subparagraph}
{\normalsize\itshape}{\thesubparagraph.}{1em}{}

\titlespacing*{\section}
{0pt}{0pt}{0pt}
\titlespacing*{\subsection}
{0pt}{0pt}{0pt}
\titlespacing*{\subsubsection}
{0pt}{0pt}{0pt}
\titlespacing*{\paragraph}
{0pt}{0pt}{0pt}
\titlespacing*{\subparagraph}
{0pt}{0pt}{0pt}

%%%%%% NO MODIFICAR %%%%%%% NO MODIFICAR %%%%%%%%%% NO MODIFICAR %%%%%%%%% NO MODIFICAR %%%%%%%%
%%%%%%%%%%%%%%% NO TOCAR %%%%%%%%%%%%% NO TOCAR %%%%%%%%%%%%% NO TOCAR %%%%%%%%%% NO TOCAR %%%%%
%%%%%% NO MODIFICAR %%%%%%%% NO MODIFICAR %%%%%%%%%% NO MODIFICAR %%%%%%% NO MODIFICAR %%%%%%%%%
%%%%%%%%%%% NO TOCA R%%%%%%%%%%%%% NO TOCAR %%%%%%%%%%%%% NO TOCAR %%%%%%%%%%%%%% NO TOCAR %%%%%

%--------Formato de Figuras----------
\graphicspath{ {Figuras/} }     % Buscar figuras en la carpeta Figuras

                    
\captionsetup[figure]           % Formato de figura
{labelfont={bf},labelformat={default},labelsep=newline,name={Figura}}

    \newcommand{\figura}[4][width=\textwidth]       % comando figura
    {
        \begin{figure}[!htb]                        % si la fig se mueve usar [h]
            \centering                              % centrado
            \caption{#3}                   % titulo de la fig
            \includegraphics[#1]{#2}                % figura
            \par
            \centering{\textbf{Fuente:} #4}                  % fuente
            \label{fig:#2}                          % referencia
        \end{figure}
    }
    
\begin{comment}

\begin{figure}[hpt]
    \centering
    
    % Título de figura
    \caption{Título Figura}
        % imagen 1
        %           Título                              Tamaño              nombre img
        \subfloat[titulo de img 1]{\includegraphics[width=0.4\columnwidth]{nombre_img_1}}
        
        % separaciones | agregar una de las opciones entre cada par de imágenes
            \qquad      % figuras en la misma linea
            \par        % siguiente línea
            
        % imagen 2
        %           Título                              Tamaño              nombre img
        \subfloat[título de img 2]{\includegraphics[width=0.4\columnwidth]{nombre_img_2}}
        
        % Se pueden agregar cuantas imágenes hagan falta
    
    %                           fuente
    \centering{\textbf{Fuente:} Fuente de la información}
    
    %               referencia
    \label{fig:nombre_de_referencia}
\end{figure}

\end{comment}

%%%%%% NO MODIFICAR %%%%%%% NO MODIFICAR %%%%%%%%%% NO MODIFICAR %%%%%%%%% NO MODIFICAR %%%%%%%%
%%%%%%%%%%%%%%% NO TOCAR %%%%%%%%%%%%% NO TOCAR %%%%%%%%%%%%% NO TOCAR %%%%%%%%%% NO TOCAR %%%%%
%%%%%% NO MODIFICAR %%%%%%%% NO MODIFICAR %%%%%%%%%% NO MODIFICAR %%%%%%% NO MODIFICAR %%%%%%%%%
%%%%%%%%%%% NO TOCA R%%%%%%%%%%%%% NO TOCAR %%%%%%%%%%%%% NO TOCAR %%%%%%%%%%%%%% NO TOCAR %%%%%

%--------Formato de Ecuaciones----------

%\captionsetup[equation]           % Formato de ecuación
%{labelfont={bf},labelformat={default},labelsep=newline,name={Figura}}

\addtolength{\belowdisplayskip}{-20pt} \setlength{\belowdisplayshortskip}{0pt}
\addtolength{\abovedisplayskip}{-20pt} \setlength{\abovedisplayshortskip}{0pt}  % Espacio superior e
                                                                                % Inferior a ecuación
\setlength{\jot}{5mm}                       % Espacio entre ecuaciones en split y align

\newenvironment{eq}[1][]
    { \begin{equation} 
        \def\temp{#1}\ifx\temp\empty
          %<EMPTY>%
        \else
            \label{eqn:#1}
        \fi
    }
    { \end{equation} }

%%%%%% NO MODIFICAR %%%%%%% NO MODIFICAR %%%%%%%%%% NO MODIFICAR %%%%%%%%% NO MODIFICAR %%%%%%%%
%%%%%%%%%%%%%%% NO TOCAR %%%%%%%%%%%%% NO TOCAR %%%%%%%%%%%%% NO TOCAR %%%%%%%%%% NO TOCAR %%%%%
%%%%%% NO MODIFICAR %%%%%%%% NO MODIFICAR %%%%%%%%%% NO MODIFICAR %%%%%%% NO MODIFICAR %%%%%%%%%
%%%%%%%%%%% NO TOCA R%%%%%%%%%%%%% NO TOCAR %%%%%%%%%%%%% NO TOCAR %%%%%%%%%%%%%% NO TOCAR %%%%%

%--------Formato de Tablas----------

\captionsetup[table]{labelfont={bf},labelformat=simple,labelsep=newline,name={Tabla}}

\newenvironment{tabla}[3][]
    { \begin{table}[!htb]
      \begin{center}
      \caption{#2}
            \ifthenelse{\isempty{#1}}%
            {\label{tab:#2}}           % if #1 is empty
            {\label{tab:#1}}           % if #1 is not empty
      \pushQED{\textbf{Fuente:} #3}   }
    { \par
      \vspace{2mm}
      \popQED
      \end{center} 
      \end{table}}
    
%------------------------------------

    \newcommand{\fig}[1]{\hyperref[fig:#1]{Figura \ref*{fig:#1}}}
    \newcommand{\eqn}[1]{\hyperref[eqn:#1]{ecuación (\ref*{eqn:#1})}}
    \newcommand{\tab}[1]{\hyperref[tab:#1]{Tabla \ref*{tab:#1}}}
    \newcommand{\anx}[1]{\hyperref[anx:#1]{\ref*{anx:#1}}}
    \newcommand{\apx}[1]{\hyperref[apx:#1]{\ref*{apx:#1}}}
    
%-------------------------------------

% Reduce itemize space
\let\olditemize\itemize
\def\itemize{\olditemize\itemsep=-1pt }
\setlist[itemize]{noitemsep, topsep=-1pt}

% Evita que imágenes menores a 80% del total de la hoja se queden solas
\renewcommand{\floatpagefraction}{.8}%

%%%%%% NO MODIFICAR %%%%%%% NO MODIFICAR %%%%%%%%%% NO MODIFICAR %%%%%%%%% NO MODIFICAR %%%%%%%%
%%%%%%%%%%%%%%% NO TOCAR %%%%%%%%%%%%% NO TOCAR %%%%%%%%%%%%% NO TOCAR %%%%%%%%%% NO TOCAR %%%%%
%%%%%% NO MODIFICAR %%%%%%%% NO MODIFICAR %%%%%%%%%% NO MODIFICAR %%%%%%% NO MODIFICAR %%%%%%%%%
%%%%%%%%%%% NO TOCA R%%%%%%%%%%%%% NO TOCAR %%%%%%%%%%%%% NO TOCAR %%%%%%%%%%%%%% NO TOCAR %%%%%

%---------------------Formato de glosario-------------------------

% \newacronym{gcd}{GCD}{Greatest Common Divisor}
% \acrlong{gcd} -> Greatest Common Divisor
% \acrshort{gcd} -> GCD
% \acrfull{lcm} -> Greatest Common Divisor (GCD)

%\setglossarystyle{long}% puede cambiar
\newglossary[slg]{symbols}{not}{ntn}{Symbols}   %crea el tipo symbols en los glosarios
\makeglossaries
% No modificar estas líneas de código, por favor dirigirse a ACRÓNIMOS o GLOSARIO

%------------------------
%
%       ACRÓNIMOS
%
%------------------------

\newacronym {UCB}      % Nombre
            {UCB-CBA}  % Acrónimo
            {Universidad Católica Boliviana "San Pablo", Unidad Académica Regional Cochabamba} % Texto largo

%------------------------
%
%       GLOSARIO
%
%------------------------

% ------- Lista de acrónimos
\newacronym {gcd} % Nombre
            {GCD} % Acrónimo
            {\textit{Greatest Common Divisor}} % Texto largo
            
\newacronym {PID} % Nombre
            {PID} % Acrónimo
            {Proporcional, Integral, Derivativo} % Texto largo

% ------- Lista de variables para el glosario 

\newglossaryentry{tau} % Nombre de símbolo
{
    type=symbols,   % Tipo de glosario (no cambiar a menos que se desee múltiples glosarios)
    name={\ensuremath{\tau}}, % Título (\ensuremath{} permite colocar símbolos matemáticos como nombre)
    description={Constante de tiempo para sistemas de control}, % Breve descripción
    sort=tau % Ordenamiento (combinación única de letras que sirven para ubicar el símbolo alfabéticamente en el índice)
}

\newglossaryentry{theta}
{
    type=symbols,
    name={\ensuremath{\theta}},
    description={Atraso de tiempo en continuo para sistemas de control},
    sort=theta
}

%%%%%%%%%%%%%%%%%%%%%%%%%%%%%%%%%%%%%%%%%%%%%%%%%%%%%%%%%%%%%%%%%%%%%%%%%%%
%%%%%%%%%%%%%%%%%%%%%%%%%%%%%% DATOS %%%%%%%%%%%%%%%%%%%%%%%%%%%%%%%%%%%%%%
%▼▼▼▼▼▼▼▼▼▼▼▼▼▼▼▼▼▼▼▼▼▼▼▼▼▼▼▼▼▼▼▼▼▼▼▼▼▼▼▼▼▼▼▼▼%

%-------------------------Personales--------------------------------

% Título del Tema  
\title{Torreta autónoma detectora de objetivos}                         

% Autor
\author{Alberto Jr. Gonzales Siles\\Kevin Claros\\Mauricio Davalos}            

% Tutor             
\newcommand{\tutor}{Edwin Calla}     % Si es mujer cambiar \tutor por \tutora (funciona con todo el comité)

% Relator
\newcommand{\relator}{Nombre del Relator} 

% Director de Carrera
\newcommand{\director}{Nombre del Director de Carrera}

% Rector Regional
\newcommand{\rector}{Nombre del Rector Regional}

%-------------------------Académicos--------------------------------

% Unidad Académica
\newcommand{\unidad}{Cochabamba}

% Departamento Académico
\newcommand{\departamento}{Departamento  de  Ingeniería y Ciencias Exactas}

% Carrera
\newcommand{\carrera}{Ingeniería Mecatrónica}

% Tipo de titulación
\newcommand{\titulacion}{Proyecto de curso}

% Grado Académico
\newcommand{\grado}{diseño servo mecánico}

% Fecha
\date{Noviembre de 2019}


%----------------Comentarios y Sugerencias---------------------------- 
        \begin{comment}
        
            Múltiples autores se colocan como:
                \author{Nombre 1\\Nombre 2\\Nombre 3}
                
            Tipos de titulación
                - Tesis de Grado
                - Proyecto de Grado
                - Trabajo Dirigido
            
            Departamentos
                - Departamento  de  Ingeniería  y  Ciencias  Exactas    
                - Departamento  de  Administración,  Economía  y  Finanzas
                - Departamento  de  Ciencias  Sociales  y  Humanas
                - Facultad  de  Enfermería  Elizabeth  Seton 
                - Facultad  de Teología
                
        \end{comment}
        

%▲▲▲▲▲▲▲▲▲▲▲▲▲▲▲▲▲▲▲▲▲▲▲▲▲▲▲▲▲▲▲▲▲▲▲▲▲▲▲▲▲▲▲▲▲▲▲▲▲▲▲▲▲▲▲▲▲▲▲▲▲▲▲▲▲▲▲▲▲▲▲▲▲▲▲▲▲▲▲▲▲▲▲▲▲▲▲▲▲▲▲▲▲%
%%%%%% NO MODIFICAR %%%%%%% NO MODIFICAR %%%%%%%%%% NO MODIFICAR %%%%%%%%% NO MODIFICAR %%%%%%%%
%%%%%%%%%%%%%%% NO TOCAR %%%%%%%%%%%%% NO TOCAR %%%%%%%%%%%%% NO TOCAR %%%%%%%%%% NO TOCAR %%%%%
%%%%%% NO MODIFICAR %%%%%%%% NO MODIFICAR %%%%%%%%%% NO MODIFICAR %%%%%%% NO MODIFICAR %%%%%%%%%
%%%%%%%%%%% NO TOCA R%%%%%%%%%%%%% NO TOCAR %%%%%%%%%%%%% NO TOCAR %%%%%%%%%%%%%% NO TOCAR %%%%%

%-------------------- CONTENIDO ----------------------

\begin{document}

    % Espacios de pie y encabezado    
        \fancyhf{}
        \renewcommand{\headrulewidth}{0pt}
        \renewcommand{\footrulewidth}{0pt}
        \rfoot{\thepage}
        \pagestyle{empty}       % Oculta el número de pagina


    % No es necesario realizar ninguna modificación en este archivo. 
% Si se lo realiza es bajo la responsabilidad del editor, siendo que podrá alterar el formato de la institución. 

\begin{titlepage}

    \newgeometry{left=3.5cm,right=3cm,top=3cm,bottom=2.6cm}  % El margen de la carátula es diferente

        % Salto de línea en mm
        \newcommand{\saltodelinea}{\vspace{1.65mm}}
        % Espacio entre texto
        \newcommand{\espaciado}{\vspace{1.5cm}}
        
        \renewcommand{\arraystretch}{1.5}
        \contourlength{0.1pt}       % Borrar si no se usa ´contour´
        \contournumber{10}
    
        \begin{center}          % Centrar texto
            
            
            \begin{textblock}{16}(0.1,1.55)
               
               \lsstyle
                % Universidad
                %{\fontsize{18}{6}\selectfont{\contour{black}{UNIVERSIDAD CATÓLICA BOLIVIANA "SAN PABLO"\ }}}
                {\fontsize{18}{6}\selectfont{\textbf{UNIVERSIDAD CATÓLICA BOLIVIANA "SAN PABLO"\ }}}
                \saltodelinea
               
                % Unidad
                {\fontsize{16}{6}\selectfont{\textbf{UNIDAD ACADÉMICA REGIONAL\ \MakeUppercase{\unidad}}}}
                \saltodelinea
                
                \contournumber{15}%
                
                % Departamento
                {\fontsize{14}{6}\selectfont{\textbf{\departamento}}}
                \saltodelinea
                
                % Carrera
                {\fontsize{14}{6}\selectfont{\textbf{Carrera de \carrera}}}
                \saltodelinea
            
            \end{textblock}
            
            \vspace*{2.8cm}
            \vspace{\fill}
            
            \includesvg[width=4.5cm, height=6.1cm]{UCB_logo}
            
            \vspace{\fill}
            
            
            % Título de la tesis
            {\fontsize{15}{18}\selectfont{\lsstyle{\textbf{\thetitle}}}\par}
            \saltodelinea
            
            
            \vspace{1.7cm}
            
            \contournumber{10}%
            
            
            % Tipo de titulación
            {\raggedleft{\fontsize{12}{6}\selectfont{{\textit{\titulacion \ de \grado \ en \carrera}}}}\par}
            
            \vspace{1.4cm}
            
            % Autor(es)
            {\fontsize{13.5}{6}\selectfont{\textbf{\theauthor}}}
            
            \vspace{1.3cm}
            
            % Lugar
            {\fontsize{12}{6}\selectfont{\unidad \ - Bolivia}}
            
            \vspace{1.2mm}
            
            % Fecha
            {\fontsize{12}{6}\selectfont{\thedate}}
            

        \end{center}
        
    \restoregeometry{}


\end{titlepage}


\newpage
\begin{center}
   \textbf{TRIBUNAL EXAMINADOR} 

    \begin{textblock}{5}(2.5,5)
    \noindent\rule{6.5cm}{0.4pt}
    \ifdef{\tutor}{\textbf{\\\tutor\\Profesor Guía}\par}{\textbf{\\\tutora\\Profesora Guía}\par}
    \end{textblock}
    
    \begin{textblock}{5}(8.5,5)
    \noindent\rule{6.5cm}{0.4pt}
    \ifdef{\relator}{\textbf{\\\relator\\Profesor Relator}\par}{\textbf{\\\relatora\\Profesora Relatora}\par}
    \end{textblock}
    
    \begin{textblock}{5}(2.5,9.5)
    \noindent\rule{6.5cm}{0.4pt}
    \ifdef{\director}{\textbf{\\\director\\Director de Carrera}\par}{\textbf{\\\directora\\Directora de Carrera}\par}
    \end{textblock}
    
    \begin{textblock}{5}(8.5,9.5)
    \noindent\rule{6.5cm}{0.4pt}
    \ifdef{\rector}{\textbf{\\\rector\\Rector Regional}\par}{\textbf{\\\rectora\\Rectora Regional}\par}
    \end{textblock}

\end{center}
    % No modificar estas líneas de código, por favor dirigirse a DEDICATORIA
% La parte de agradecimientos se deja a criterio de cada uno, siendo la única regla que se mantenga el tamaño de letra y formato establecido por la institución. 

\newpage

%------------------------
%
%       DEDICATORIA
%
%------------------------

% Ejemplo
\begin{flushright}
    \begin{itshape}
        \vspace*{\fill}
        
        \begin{tabular}{| J{4cm}}
            \hspace{\fill} Agradecimientos\\
            \newline
            Gracias a \LaTeX por ser una herramienta que ayuda tanto en formato, a mi persona  por ser persistente en el tema a tratar. % Escribir aquí lo que se desea
        \end{tabular}
        
        \vspace{\fill}
    \end{itshape}
\end{flushright}
    
        
    % Espaciados de hoja
        \setlength{\parindent}{0pt}     % Sangría
        \setlength{\parskip}{4mm}       % Entre párrafos
        \spacing{1.4}                   % Interlineado
    
    
    % No modificar estas líneas de código, por favor dirigirse a RESUMEN y ABSTRACT
\newpage

%------------------------
%
%       RESUMEN 
%
%------------------------

\section*{RESUMEN}
Texto pertinente en español

\textbf{Palabras clave:} sentry gun, OPENCV, Raspberry, stepper, Torreta autonoma, Reconocimiento de objetos, Object Detection, Deteccion de color,Autonomia Letal.

\newpage

%------------------------
%
%       ABSTRACT
%
%------------------------

\section*{ABSTRACT}
Texto pertinente en inglés 

\textbf{Keywords:} word 1, word 2, ....
    
    
    % Índice General
        \setcounter{tocdepth}{3}            % Profundidad de índice
        \renewcommand{\contentsname}{ÍNDICE GENERAL} 
        \addtocontents{toc}{\protect\thispagestyle{empty}}
        \newpage\tableofcontents
    
    % Índice de figuras
        \renewcommand{\listfigurename}{ÍNDICE DE FIGURAS}
        \newpage\listoffigures
        \thispagestyle{empty}
    
    % Índice de tablas                                  BORRAR SI NO SE USARA
        \renewcommand{\listtablename}{ÍNDICE DE TABLAS}
        \newpage\listoftables
        \thispagestyle{empty}

    % Lista de pseudocódigos                                  BORRAR SI NO SE USARA
        \renewcommand{\listalgorithmname}{\centerline{ÍNDICE DE ALGORITMOS}}
        \newpage\listofalgorithms
        \thispagestyle{empty}
        
    % Índice de símbolos                                    BORRAR SI NO SE USARA
        \newpage
        %\glsaddall      % BORRAR PARA SOLO MOSTRAR EN GLOSARIO VALORES UTILIZADOS
        \printglossary[type=symbols, title={\normalsize\centerline{\textbf{ÍNDICE DE SÍMBOLOS}}}]
        \thispagestyle{empty}

    %Glosario de acrónimos                                  BORRAR SI NO SE USARA
        \newpage
        \printglossary[type=\acronymtype, title={\normalsize\centerline{\textbf{GLOSARIO DE ACRÓNIMOS}}}]
        \thispagestyle{empty}

    % Estilo de hoja
        \pagestyle{fancy}

    % No modificar estas líneas de código, por favor dirigirse a INTRODUCCIÓN 
\newpage
\phantomsection\addcontentsline{toc}{section}{INTRODUCCIÓN} 

%------------------------
%
%       INTRODUCCIÓN
%
%------------------------

\section*{INTRODUCCIÓN}

La tendencia hacia sistemas autónomos de defensa en la sociedad se ha debido al énfasis en la pronta intervención a cualquier echo delictivo para reducir las víctimas humanas durante dichos conflictos. Sin embargo en la actualidad los sistemas de seguridad son mayormente utilizados en empresas y bancos, esto causa que existan mucho barrios con poca seguridad, en especial en zonas de bajos recursos. 

La motivación del proyecto es esforzarse por un sistema autónomo que pueda detectar un objetivo en movimiento y tenga la capacidad de neutralizarlo. La parte a estudiar en este proyecto sera la detección de objetivos en movimientos, es decir, que el centinela sera capaz de detectar y localizar intrusos con eficacia. Para esto el diseño de la torreta autónoma permitirá neutralizar a cualquier objetivo dentro del campo de visión mediante la utilización de visión artificial.

"Los humanos utilizan sus ojos y cerebros para ver y sentir el mundo que los rodea. La visión artificial es la ciencia dedicada a proveer una habilidad similar, o incluso mejor a las maquinas y computadoras ".\cite{barrero2015algoritmo}
"La visión artificial se preocupa con la extracción automática, análisis y entendimiento de información útil presentada en una imagen o secuencia de imágenes. Esto involucra el desarrollo de una base teórica y algorítmica para lograr un entendimiento visual."\cite{BMVA:Online}

Las cámaras son uno de los sensores más utilizados ya que estos poseen diversas aplicaciones, es decir, que le podemos dar diversos usos\cite{sankaranarayanan2008object}. Es por esto que su utilización al momento de diseñar un centinela que detecte un objetivo en un rango de visión dado es visto como una necesidad. 

Existen diversas formas de crear piezas móviles, sin embargo dadas nuestras condiciones actuales se intentara utilizar materiales que se encuentren en nuestra posesión o que sean fáciles de conseguir para el armazón de la torreta.

\textit{"The turret that is included in the system will be a paintball or airsoft gun that will simulate the real effect of being hit by a military defense turret. (...) Two servo motors will control the pan and tilt of the turret and the third will directly control the firing of the system by controlling when the trigger is pulled."} \cite{AutonomousTurret} Como se puede ver ya existe proyectos en los cuales la accesibilidad ha sido considerada y es por esto que se utilizaran dichos artículos o proyectos para basar los materiales necesarios. Igualmente dado que la visión es el enfoque principal de nuestro proyecto se intentara mantener los costos del armazón en un mínimo. 

Para poder realizar el proyecto eficientemente sera necesario seleccionar aquellos documentos que realmente proporcionen información necesaria y útil, para lo que se realizara por medio de una revisión sistemática de la literatura. Para esto primero sera necesario establecer los objetivos a que intentaremos alanzar. Después se realizara la revisión sistemática en si y se agruparan aquellos documentos que se presenten información relevante al tema. Finalmente se utilizarán los documentos elegidos y se usaran como aportes y sustentos para la elaboración de la torreta.

\subsection*{Antecedentes}

Hay muchas formas de detectar movimiento, pero particularmente en robótica, los enfoques más generales utilizan el procesamiento de imágenes o seguimiento de detección de luz y rango.
Estos dos métodos requieren básicamente un procesamiento extenso sensores de potencia o costosos. Muchas soluciones de grado militar están disponibles para esto problema pero generalmente son soluciones muy caras y una solución de bajo precio para este problema lo hace más potente. Ahora los militares están ganando mucha práctica experiencia y trabajo para la interacción de humanos y robots La robótica es una rama tecnológica en evolución que se ocupa del diseño, construcción, operación y aplicación de robots y sistemas informáticos para su control, retroalimentación sensorial y procesamiento de información. Mejora la seguridad, reduce las causas innecesarias y Reduce el costo operativo.

\subsection*{Identificación del problema}
Una gran problemática en nuestro medio es la inseguridad ciudadana y esto empieza desde el hogar y termina en la autoridades competentes, la implementación de nuevos sistemas de seguridad en los recintos penitenciarios es muy importante para dejar de lado el elemento humano al momento de tomar decisiones que tienen que ser frías y no evaluadas por un ser con sentimientos.

\subsection*{Formulación del problema}

podremos implementar un sistema de reconocimiento de objetivo a través de  la plataforma raspberry, con el apoyo de Open cv aplicado el el software stretch?

\subsection*{Objetivos}

desarrollar un prototipo de seguimiento de objetivo con la implementación de la plataforma raspberry y la pi cámara con ayuda de Open CV para la detección de objetivos en movimiento.

\subsubsection*{Objetivo general}

Implementar el prototipo con éxito.

\subsubsection*{Objetivos específicos}

se realizaran las siguientes tareas:
\begin{itemize}
\item instalar con éxito open CV en una plataforma pi zero, de no conseguirlo usar las plataformas pi 3 o superiores .
\item construir una estructura acorde con el movimiento generado de búsqueda.
\item aplicar los conocimientos de la materia de diseño servomecanico. 
\end{itemize}


\subsection*{Hipótesis}

El hecho de tener una gran cantidad de plataformas de procesamiento de señal nos deja la duda, de cual es mas óptima sin necesidad de gastar innecesariamente en una plataforma costosa, además de usar los motores adecuados al momento de generar los controles básicos 


\subsection*{Justificación}

el tema fue elegido gracias a la tendencia de que la inseguridad cada ves es mas notoria y la robótica pude ser una gran solución al momento de tomar decisiones frías y que protejan el interés de la ciudadanía dejando de lado la leyes de la ROBÓTICA e infundiendo castigo severo a reos que incumplan con normas de seguridad de prisiones en la ciudad de Cochabamba. 

\subsection*{Materiales}
Ya que este es un prototipo, se emplearan los siguientes materiales económicos:
\begin{itemize}
\item Raspberry pi zero ó raspberry pi 3b+
\item pi camera ó web cam.
\item 2 stepper nema 17.
\item marcadora de paintball ó replica de airsoft.
\item Single Relay.
\item Step Up Converter.
\item Adafruit TB6612 Motion Motor Control Shield Board.
\item estructura para marcadora ó replica.

\end{itemize}


\subsection*{Cronograma}
el siguiente cronograma (tabla 1) se aplicara desde la segunda semana de octubre y son operaciones generales.
% Please add the following required packages to your document preamble:
% \usepackage[table,xcdraw]{xcolor}
% If you use beamer only pass "xcolor=table" option, i.e. \documentclass[xcolor=table]{beamer}
\begin{table}[]
\centering
\begin{tabular}{|l|l|l|l|l|l|l|l|l|l|}
\hline
\# & objetivo & \multicolumn{4}{l|}{octubre} & \multicolumn{4}{l|}{noviembre} \\ \hline
1 & instalacion de software &  & \cellcolor[HTML]{34FF34} & \cellcolor[HTML]{34FF34} &  &  &  &  &  \\ \hline
2 & armado de estructura &  &  & \cellcolor[HTML]{34FF34} & \cellcolor[HTML]{34FF34} & \cellcolor[HTML]{34FF34} &  &  &  \\ \hline
3 & implementacion de drivers &  &  & \cellcolor[HTML]{34FF34} & \cellcolor[HTML]{34FF34} & \cellcolor[HTML]{34FF34} &  &  &  \\ \hline
4 & ultimar detalles &  &  &  &  &  & \cellcolor[HTML]{34FF34} &  &  \\ \hline
5 & presentar &  &  &  &  &  &  & \cellcolor[HTML]{34FF34} & \cellcolor[HTML]{34FF34} \\ \hline
\end{tabular}
\caption{}
\label{tab:my-table}
\end{table}
    % No modificar estas líneas de código, por favor dirigirse a MARCO TEÓRICO

\newpage

%------------------------
%
%       MARCO TEÓRICO
%
%------------------------

\section{MARCO TEÓRICO}

    % No modificar estas líneas de código, por favor dirigirse a MARCO METODOLÓGICO

\newpage

%------------------------
%
%       MARCO METODOLÓGICO 
%
%------------------------

\section{MARCO METODOLÓGICO}

    % No modificar estas líneas de código, por favor dirigirse a MARCO PRÁCTICO 

\newpage

%------------------------
%
%       MARCO PRÁCTICO
%
%------------------------

\section{MARCO PRÁCTICO}

    % No modificar estas líneas de código, por favor dirigirse a CONCLUSIONES

\newpage
\phantomsection\addcontentsline{toc}{section}{CONCLUSIONES} 

%------------------------
%
%       CONCLUSIONES
%
%------------------------

\section*{CONCLUSIONES}
En el proyecto físico igualmente pudimos observar que si bien la plataforma de Raspberry es inmensamente útil esta no es muy amigable, dado que la programación se tuvo que realizar en lenguaje Python, el cual no había sido utilizado en nuestra carrera hasta el momento. Igualmente si bien la librería OpenCV(\textit{Open Computer Vision}) resulto útil gracias a su gran cantidad de procesos existentes también resulto ser difícil de implementar dado su gran tamaño y su tiempo de compilación. Esto ocasiono un retraso y un incremento en el costo ya que se tuvo que reemplazar la memoria que estaba siendo utilizada por una de mayor capacidad.
    % No modificar estas líneas de código, por favor dirigirse a RECOMENDACIONES

\newpage
\phantomsection\addcontentsline{toc}{section}{RECOMENDACIONES} 

%------------------------
%
%       RECOMENDACIONES
%
%------------------------

\section*{RECOMENDACIONES}
Introducir lo que se desea....

    % Bibliografía
        \newpage
        \begingroup
        \setstretch{0.8}
        \setlength{\bibhang}{0pt}
        \setlength\bibitemsep{12pt}
%        \nocite{*}  % Adicionar para mostrar mostrar todas las bibliografías siempre
        \printbibliography[heading=bibintoc,title={BIBLIOGRAFÍA}]
        \endgroup

    % No modificar estas líneas de código, por favor dirigirse a ANEXOS más abajo.

            \newpage
            \addtocontents{toc}{\cftpagenumbersoff{section}} 
            \phantomsection\addcontentsline{toc}{section}{ANEXOS}
            \addtocontents{toc}{\cftpagenumberson{section}} 
            
            \newcommand{\anex}[1]
            {
                \newpage
                \vspace*{\fill}
                \section[\normalfont#1]{#1}
                \label{anx:#1}
                \vspace*{\fill}
                \newpage
            }
            
            \appendix
            \renewcommand{\thesection}{\normalfont Anexo \arabic{section}}
            \addtocontents{toc}{\setlength{\cftsecnumwidth}{1.8cm}}
            \addtocontents{toc}{\protect\setstretch{1.2}}
            \addtocontents{toc}{\cftsecpagefont}{}
            
            \setcounter{page}{1}
            \setcounter{figure}{0}
            \setcounter{table}{0}
            \titleformat{\section}{\bfseries\normalsize\flushright}{\MakeUppercase{\thesection}}{0em}{\\ \MakeUppercase}
            \captionsetup{list=no} % Para que no aparezca en índice general 

%------------------------
%
%       ANEXOS
%
%------------------------

\anex{Título de Anexo 1}
Introducir lo que se desea.... ahora aprovecharemos de dejar indicaciones para utilizar correctamente anexos y apéndices. 

Los comandos \verb|\anex{}| y \verb|\apex{}| crean una hoja de anexo o apéndice con el título que se le asigne. 

Para referenciar anexos o apéndices existen los comandos \verb|\anx{}| y \verb|\apx{}| que funcionan con el mismo título del anexo o apéndice.

Entre otros comandos de mucha ayuda para lo que es anexos y apéndices, es el poder adicionar documentos en formato PDF de forma directa a nuestro \LaTeX, esto podremos realizar con el comando:

\verb|\includepdf[pages={1,2,n pag del pdf}]{PDFs/nombre_archivo.pdf}|. 

El resultado de esto será el presentado en el anexo a continuación.

\anex{Título de Anexo 2}
\includepdf[pages={1}]{PDFs/Google_Academic.pdf}
    % No modificar estas líneas de código, por favor dirigirse a APÉNDICE más abajo.

            \newpage
            \addtocontents{toc}{\cftpagenumbersoff{section}} 
            \phantomsection\addcontentsline{toc}{section}{APÉNDICES}
            \addtocontents{toc}{\cftpagenumberson{section}} 
            
            \newcommand{\apex}[1]
            {
                \newpage
                \vspace*{\fill}
                \section[\normalfont#1]{#1}
                \label{apx:#1}
                \vspace*{\fill}
                \newpage
            }
            
            \appendix
            \renewcommand{\thesection}{\normalfont Apéndice \arabic{section}}
            \addtocontents{toc}{\setlength{\cftsecnumwidth}{2.3cm}}
            \addtocontents{toc}{\protect\setstretch{1.2}}
            \addtocontents{toc}{\cftsecpagefont}{}
            
            \setcounter{page}{1}
            \setcounter{figure}{0}
            \setcounter{table}{0}
            \titleformat{\section}{\bfseries\normalsize\flushright}{\MakeUppercase{\thesection}}{0em}{\\ \MakeUppercase}
            \captionsetup{list=no} % Para que no aparezca en índice general 
            
%------------------------
%
%       APÉNDICE
%
%------------------------

\apex{Título de Apéndice 1}
Introducir lo que se desea... aprovecharemos este espacio para explicar como utilizar el glosario. 

Para introducir un nuevo acrónimo, nos debemos dirigir a \verb|Glosario.tex| y adicionamos la siguiente línea:

\verb|\newacronym{Identificador o label}{Acrónimo}{Acrónimo desglosado}|

Un ejemplo sería:
\verb|\newacronym{gcd}{GCD}{Greatest Common Divisor}|


Para hacer referencia al acrónimo se utiliza los comandos \verb|\acrlong{}|, \verb|\acrshort{}| o \verb|\acrfull{}|, generando las siguientes maneras de representar respectivamente:

\begin{itemize}
    \item \acrlong{gcd}
    \item \acrshort{gcd}
    \item \acrfull{gcd}
\end{itemize}

Para adicionar un nuevo símbolo o variable, nos dirigimos nuevamente a \verb|Glosario.tex| y adicionamos de la siguiente manera:

\begin{verbatim}
\newglossaryentry{nombre}
{
    type=symbols,
    name={t},
    description={something},
    sort=t
}
\end{verbatim}

Para utilizar dichas variables, se utiliza el comando \verb|\gls{}| de la siguiente manera:
 
\begin{itemize}
    \item \verb|\gls{theta}| resultando: \gls{theta} 
    \item \verb|\gls{tau}| resultando: \gls{tau}
\end{itemize}

De todas formas, para mayor detalle de como usar dichos comandos, usted puede dirigirse al archivo \verb|Glosario.tex|, donde ya se encuentran ejemplos debidamente comentados para su fácil entendimiento.

\apex{Título de Apéndice 2}
Introducir lo que se dese...

Se aprovechará el presente apéndice para dejar un ejemplo de como realizar un pseudocódigo en caso de que fuera necesario para cualquier tipo de \textit{software} u otros. 

\begin{algorithm}[ht!]
	\caption{Título del pseudo algoritmo}
	\label{Pseudo_alg} % Etiqueta con la que se va referenciar 
	\begin{algorithmic}[1] 
		\REQUIRE $ $
			\STATE Inicializar variables necesarias
			\STATE Índices de desempeño IAE, ITAE y TVC
			\STATE \For{$k=1\dots nit$}
			{
			    \hspace{0.5cm} a) Leer señal de salida \\
				\hspace{0.5cm} b) Calcular el error \\
				\hspace{0.5cm} c) Calcular señal de control \\
				\hspace{0.9cm} $u(k)$ \\
				\hspace{0.5cm} d) Enviar señal de control \\
				\hspace{0.5cm} e) Calcular índices de desempeño
			}
			\STATE Plotear resultados
			\STATE Mostrar índices de desempeño
			\ENSURE $ $
	\end{algorithmic}
\end{algorithm}

Como podemos apreciar, el código tiene la misma estructura que venimos trabajando tanto en figuras, tablas y ecuaciones. Igualmente tenemos la facilidad de utilizar referencias mediante la etiqueta y que se genere una lista de algoritmos automáticamente. 

Caso usted quisiera adicionar pequeñas líneas de código tal cual fueron introducidas por su persona, puede utilizar el formato a continuación.

\begin{lstlisting}[frame=single]
  % suma de los elementos de un vector
  z = 0;
  n = length(v);
  for i=1:1:n
    z = z + v[i];
  end
\end{lstlisting}
    
\end{document}