% No modificar estas líneas de código, por favor dirigirse a ANEXOS más abajo.

            \newpage
            \addtocontents{toc}{\cftpagenumbersoff{section}} 
            \phantomsection\addcontentsline{toc}{section}{ANEXOS}
            \addtocontents{toc}{\cftpagenumberson{section}} 
            
            \newcommand{\anex}[1]
            {
                \newpage
                \vspace*{\fill}
                \section[\normalfont#1]{#1}
                \label{anx:#1}
                \vspace*{\fill}
                \newpage
            }
            
            \appendix
            \renewcommand{\thesection}{\normalfont Anexo \arabic{section}}
            \addtocontents{toc}{\setlength{\cftsecnumwidth}{1.8cm}}
            \addtocontents{toc}{\protect\setstretch{1.2}}
            \addtocontents{toc}{\cftsecpagefont}{}
            
            \setcounter{page}{1}
            \setcounter{figure}{0}
            \setcounter{table}{0}
            \titleformat{\section}{\bfseries\normalsize\flushright}{\MakeUppercase{\thesection}}{0em}{\\ \MakeUppercase}
            \captionsetup{list=no} % Para que no aparezca en índice general 

%------------------------
%
%       ANEXOS
%
%------------------------

\anex{Título de Anexo 1}
Introducir lo que se desea.... ahora aprovecharemos de dejar indicaciones para utilizar correctamente anexos y apéndices. 

Los comandos \verb|\anex{}| y \verb|\apex{}| crean una hoja de anexo o apéndice con el título que se le asigne. 

Para referenciar anexos o apéndices existen los comandos \verb|\anx{}| y \verb|\apx{}| que funcionan con el mismo título del anexo o apéndice.

Entre otros comandos de mucha ayuda para lo que es anexos y apéndices, es el poder adicionar documentos en formato PDF de forma directa a nuestro \LaTeX, esto podremos realizar con el comando:

\verb|\includepdf[pages={1,2,n pag del pdf}]{PDFs/nombre_archivo.pdf}|. 

El resultado de esto será el presentado en el anexo a continuación.

\anex{Título de Anexo 2}
\includepdf[pages={1}]{PDFs/Google_Academic.pdf}