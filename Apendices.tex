% No modificar estas líneas de código, por favor dirigirse a APÉNDICE más abajo.

            \newpage
            \addtocontents{toc}{\cftpagenumbersoff{section}} 
            \phantomsection\addcontentsline{toc}{section}{APÉNDICES}
            \addtocontents{toc}{\cftpagenumberson{section}} 
            
            \newcommand{\apex}[1]
            {
                \newpage
                \vspace*{\fill}
                \section[\normalfont#1]{#1}
                \label{apx:#1}
                \vspace*{\fill}
                \newpage
            }
            
            \appendix
            \renewcommand{\thesection}{\normalfont Apéndice \arabic{section}}
            \addtocontents{toc}{\setlength{\cftsecnumwidth}{2.3cm}}
            \addtocontents{toc}{\protect\setstretch{1.2}}
            \addtocontents{toc}{\cftsecpagefont}{}
            
            \setcounter{page}{1}
            \setcounter{figure}{0}
            \setcounter{table}{0}
            \titleformat{\section}{\bfseries\normalsize\flushright}{\MakeUppercase{\thesection}}{0em}{\\ \MakeUppercase}
            \captionsetup{list=no} % Para que no aparezca en índice general 
            
%------------------------
%
%       APÉNDICE
%
%------------------------

\apex{Título de Apéndice 1}
Introducir lo que se desea... aprovecharemos este espacio para explicar como utilizar el glosario. 

Para introducir un nuevo acrónimo, nos debemos dirigir a \verb|Glosario.tex| y adicionamos la siguiente línea:

\verb|\newacronym{Identificador o label}{Acrónimo}{Acrónimo desglosado}|

Un ejemplo sería:
\verb|\newacronym{gcd}{GCD}{Greatest Common Divisor}|


Para hacer referencia al acrónimo se utiliza los comandos \verb|\acrlong{}|, \verb|\acrshort{}| o \verb|\acrfull{}|, generando las siguientes maneras de representar respectivamente:

\begin{itemize}
    \item \acrlong{gcd}
    \item \acrshort{gcd}
    \item \acrfull{gcd}
\end{itemize}

Para adicionar un nuevo símbolo o variable, nos dirigimos nuevamente a \verb|Glosario.tex| y adicionamos de la siguiente manera:

\begin{verbatim}
\newglossaryentry{nombre}
{
    type=symbols,
    name={t},
    description={something},
    sort=t
}
\end{verbatim}

Para utilizar dichas variables, se utiliza el comando \verb|\gls{}| de la siguiente manera:
 
\begin{itemize}
    \item \verb|\gls{theta}| resultando: \gls{theta} 
    \item \verb|\gls{tau}| resultando: \gls{tau}
\end{itemize}

De todas formas, para mayor detalle de como usar dichos comandos, usted puede dirigirse al archivo \verb|Glosario.tex|, donde ya se encuentran ejemplos debidamente comentados para su fácil entendimiento.

\apex{Título de Apéndice 2}
Introducir lo que se dese...

Se aprovechará el presente apéndice para dejar un ejemplo de como realizar un pseudocódigo en caso de que fuera necesario para cualquier tipo de \textit{software} u otros. 

\begin{algorithm}[ht!]
	\caption{Título del pseudo algoritmo}
	\label{Pseudo_alg} % Etiqueta con la que se va referenciar 
	\begin{algorithmic}[1] 
		\REQUIRE $ $
			\STATE Inicializar variables necesarias
			\STATE Índices de desempeño IAE, ITAE y TVC
			\STATE \For{$k=1\dots nit$}
			{
			    \hspace{0.5cm} a) Leer señal de salida \\
				\hspace{0.5cm} b) Calcular el error \\
				\hspace{0.5cm} c) Calcular señal de control \\
				\hspace{0.9cm} $u(k)$ \\
				\hspace{0.5cm} d) Enviar señal de control \\
				\hspace{0.5cm} e) Calcular índices de desempeño
			}
			\STATE Plotear resultados
			\STATE Mostrar índices de desempeño
			\ENSURE $ $
	\end{algorithmic}
\end{algorithm}

Como podemos apreciar, el código tiene la misma estructura que venimos trabajando tanto en figuras, tablas y ecuaciones. Igualmente tenemos la facilidad de utilizar referencias mediante la etiqueta y que se genere una lista de algoritmos automáticamente. 

Caso usted quisiera adicionar pequeñas líneas de código tal cual fueron introducidas por su persona, puede utilizar el formato a continuación.

\begin{lstlisting}[frame=single]
  % suma de los elementos de un vector
  z = 0;
  n = length(v);
  for i=1:1:n
    z = z + v[i];
  end
\end{lstlisting}